\section{Introducción}
Esta es la introducción
\subsection{Objetivos}
Texto grande:
\begin{huge}
Hacer una demostración.
\end{huge}

\subsubsection{Listas}
Lista con viñetas:
\begin{itemize}
\item Ob1.
\item Ob2.
\end{itemize}

Lista numerada:
\begin{enumerate}
\item No 1.
\item No 2.
\end{enumerate}

\section*{Sección no numerada}
Fin de sección.

\section{Matemática básica}

\subsection{Fundamentos}
Símbolos entre el texto: $a^2 + b^2 = c^2$.

Función definida por casos:
\begin{equation}
f(x) = %
\begin{cases}
	1	& \text{Si } x > 0 \\
	-1	& \text{Si } x < 0 \\
	0   & \text{Si } x = 0
\end{cases}
\end{equation}

Potencias y raíces:
\begin{equation}
\sqrt{b^2 - 4ac}
\end{equation}

Fracciones
\begin{equation}
\frac{a^2}{(b + c)^2} + \frac{1}{n}
\end{equation}

Agrupadores de tamaño variable:
\begin{equation}
\Big[ \Big( \frac{1}{n} + x^2 \Big) - k\big(x + 1\big)^3 \Big]
\end{equation}

Matrices (como ecuación no numerada):
\begin{equation*}
 \left(
    \begin{array}{ccc}
        a & b & c \\
        d & e & f \\
        g & h & i
    \end{array}
\right)
\end{equation*}

Ambiente ``ecuación'' con sumatoria (promedio):
\begin{equation}
\bar{x} = \frac{1}{n} \sum_{i=1}^{n} x_i
\label{media}
\end{equation}

Ambiente ``ecuación'' con integral:
\begin{equation}
\int _a ^b f(x) dx
\end{equation}

\subsection{Ejemplos de símbolos y constantes}

Volumen de una esfera:
\begin{equation}
\frac{4}{3} \pi r^3
\end{equation}

Regla trapezoidal uniforme:
\begin{equation}
\int_a ^b f(x)dx \approx \sum_{k=1}{N} \big( f(x_{k+1}) + f(x_k) \big)
\end{equation}

Varianza:
\begin{equation}
\sigma^2 = \frac{1}{N} \sum_{i=1}^{N} (x_i - \mu)^2
\end{equation}

donde $\mu$ es la media definida en \ref{media}.
\newline

Y otros símbolos:
\begin{itemize}
\item Derivadas parciales: $\dfrac{\partial f}{\partial x}$

\item Letras griegas: $\alpha, \beta, \gamma, \Gamma, \delta, \Delta, \sigma, \Sigma, \ldots$.

\item Fórmulas: $\forall x\exists y ~|~ x \in C \land y \in A \cap B$, $\nexists n ~|~ N \cup \{n\} = \emptyset$

\item Símbolos adicionales: $\mathbb{Z}, \mathbb{N}, \mathbb{R}$

\item Y más: $\mathscr{A}, \mathscr{B}, \mathscr{Z}$
\end{itemize}

\section{Imágenes y figuras}


Figura centrada:
\begin{figure}[ht]
\centering
\includegraphics[scale=0.8]{imgL1/cosme.jpg}
\caption{Cosme Fulanito}
\label{fig:cosme}
\end{figure}
La figura \ref{fig:cosme} muestra al destacado autor de este trabajo.

Imagen a la derecha:

\begin{flushright}
\includegraphics[scale=0.5]{imgL1/cosme.jpg}
\end{flushright}




\section{Tablas}
Tablita:

\begin{table}[ht]
\begin{center}
    \begin{tabular}{ c | c | c }
        1 & 2 & 3 \\ 
        \hline 
        4 & 5 & 6 \\  
        7 & 8 & 9    
    \end{tabular}
\end{center}
\caption{Tabla pequeña}
\end{table}
tabla larga partida en paginas:\\


\begin{longtable}{|c|c|c|c|}
        \hline
        \textbf{$\Delta \tau$ (píxeles)} & \textbf{$\Delta \rho$ (píxeles)} & \textbf{$\Delta \phi$ (grados)} & \textbf{Clasificador} \\  
        \hline
        \endhead
        1 & 1 & 0,5 & K-Means \\ \hline
        1 & 1 & 0,5 & SMO \\ \hline
        1 & 1 & 0,5 & LogitBoost \\ \hline
        \ldots & ... & ... & ... \\ \hline
        1 & 1 & 1 & K-Means \\ \hline
        1 & 1 & 1 & SMO \\ \hline
        ... & ... & ... & ... \\ \hline
        1 & 1 & 2 & K-Means \\ \hline
        1 & 1 & 2 & SMO \\ \hline
        ... & ... & ... & ... \\ \hline
        3 & 3 & 2 & BayesNet \\ \hline
        1 & 1 & 0,5 & K-Means \\ \hline
        1 & 1 & 0,5 & SMO \\ \hline
        1 & 1 & 0,5 & LogitBoost \\ \hline
        ... & ... & ... & ... \\ \hline
        1 & 1 & 1 & K-Means \\ \hline
        1 & 1 & 1 & SMO \\ \hline
        ... & ... & ... & ... \\ \hline
        1 & 1 & 2 & K-Means \\ \hline
        1 & 1 & 2 & SMO \\ \hline
        ... & ... & ... & ... \\ \hline
        3 & 3 & 2 & BayesNet \\ \hline
        1 & 1 & 0,5 & K-Means \\ \hline
        1 & 1 & 0,5 & SMO \\ \hline
        1 & 1 & 0,5 & LogitBoost \\ \hline
        ... & ... & ... & ... \\ \hline
        1 & 1 & 1 & K-Means \\ \hline
        1 & 1 & 1 & SMO \\ \hline
        ... & ... & ... & ... \\ \hline
        1 & 1 & 2 & K-Means \\ \hline
        1 & 1 & 2 & SMO \\ \hline
        ... & ... & ... & ... \\ \hline
        3 & 3 & 2 & BayesNet \\ \hline
        1 & 1 & 1 & K-Means \\ \hline
        1 & 1 & 1 & SMO \\ \hline
        ... & ... & ... & ... \\ \hline
        1 & 1 & 2 & K-Means \\ \hline
        1 & 1 & 2 & SMO \\ \hline
        ... & ... & ... & ... \\ \hline
        3 & 3 & 2 & BayesNet \\ \hline
        1 & 1 & 0,5 & K-Means \\ \hline
        1 & 1 & 0,5 & SMO \\ \hline
        1 & 1 & 0,5 & LogitBoost \\ \hline
        ... & ... & ... & ... \\ \hline
        1 & 1 & 1 & K-Means \\ \hline
        1 & 1 & 1 & SMO \\ \hline
        ... & ... & ... & ... \\ \hline
        1 & 1 & 2 & K-Means \\ \hline
        1 & 1 & 2 & SMO \\ \hline
        ... & ... & ... & ... \\ \hline
        3 & 3 & 2 & BayesNet \\ \hline
        1 & 1 & 1 & K-Means \\ \hline
        1 & 1 & 1 & SMO \\ \hline
        ... & ... & ... & ... \\ \hline
        1 & 1 & 2 & K-Means \\ \hline
        1 & 1 & 2 & SMO \\ \hline
        ... & ... & ... & ... \\ \hline
        3 & 3 & 2 & BayesNet \\ \hline
        1 & 1 & 0,5 & K-Means \\ \hline
        1 & 1 & 0,5 & SMO \\ \hline
        1 & 1 & 0,5 & LogitBoost \\ \hline
        ... & ... & ... & ... \\ \hline
        1 & 1 & 1 & K-Means \\ \hline
        1 & 1 & 1 & SMO \\ \hline
        ... & ... & ... & ... \\ \hline
        1 & 1 & 2 & K-Means \\ \hline
        1 & 1 & 2 & SMO \\ \hline
        ... & ... & ... & ... \\ \hline
        3 & 3 & 2 & BayesNet \\ \hline
        
    
    \caption{Grupos estudiados en el experimento B.}
    \label{tab:gruposB}
\end{longtable}

\section{Uso de referencias bibliográficas}
Para citar trabajos se utiliza la instrucción \textbf{cite} \cite{Bonaparte75}.

Dos o más se pueden separar por coma \cite{Brown, Fulanito1999}.